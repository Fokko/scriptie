% the abstract
The growth of information in today’s society is clearly exponential. To process these staggering amounts of data, the classical approaches are not up to the task. Instead we need highly parallel software running on tens, hundreds, or even thousands of servers to process the data. This research presents an introduction into outlier detection and its applications. An outlier is one or multiple observations that deviates quantitatively from the majority of the dataset and may be the subject of further investigation. After comparing different approaches to outlier detection, a scalable implementation of the unsupervised Stochastic Outlier Selection algorithm is given. The Docker-based microservice architecture allows dynamically scaling according to the current needs. The application stack consists of Apache Spark as the computational engine, Apache Kafka as data store and Apache Zookeeper as synchronization service to ensure high reliability. Based on this we empirically observe the quadratic time complexity of the algorithm as expected. We explore the importance of matching the number of Spark worker nodes to the available CPU of the underlying hardware to maximize the performance. Finally the effect of the distributed data-shuffles is discussed which is sometimes necessary for synchronizing data between the different worker nodes. Although some optimizations can be done, this research presents a scalable and highly available outlier detection system which can be used to detect outliers on streams of data.